\documentclass[aspectratio=169]{beamer}

\usepackage{graphicx}

\usepackage{xcolor}

\definecolor{DarkGreen}{rgb}{0.0, 0.4, 0.23} % not so dark

\usepackage{amsfonts}
\newcommand{\tick}{\checkmark}
\newcommand{\cross}{$\times$}


\newcommand{\myhref}[2]{\href{#2}{#1}}

%\setbeamertemplate{footline}[page number]
\newcommand{\setpresentershort}[1]{\setbeamertemplate {footline}{\tiny{}#1\insertpagenumber/\pageref{LastPage}
    %\strut\quad{}Peter Sewell\quad{}Underpinning the foundations
    \hfill}}
\newcommand{\setpresenter}[1]{\setbeamertemplate {footline}{\tiny{}#1\insertpagenumber/\pageref{LastPage}\strut%\quad{}Peter Sewell\quad{}Underpinning the foundations
    \hfill}}
%  \inserttotalframenumber
%\setbeamertemplate {footline}{\quad\hfill\insertframenumber\strut\quad}
%\setpresenter{Peter}
\setpresenter{}

\beamertemplatenavigationsymbolsempty
\setbeamersize{text margin left=5mm,text margin right=5mm} 
\setbeamertemplate{frametitle}[default][center]

\hypersetup{colorlinks=true,linkcolor=blue,urlcolor=blue}






\usepackage[T1]{fontenc}


\newenvironment{tightenumerate}{
\begin{enumerate}
  \setlength{\itemsep}{1pt}
  \setlength{\parskip}{0pt}
  \setlength{\parsep}{0pt}}{\end{enumerate}
}
\newenvironment{tightitemize}{
\begin{itemize}
  \setlength{\itemsep}{1pt}
  \setlength{\parskip}{0pt}
  \setlength{\parsep}{0pt}}{\end{itemize}
}


\definecolor{mc}{rgb}{0.0,0.35,0.35}
\definecolor{emc}{rgb}{0.9,0.0,0.4}
\newcommand{\mc}{\color{mc}}
\newcommand{\emc}{\color{emc}}


\usepackage{listings}
\lstset{showstringspaces=false}

%\lstset{basicstyle=\footnotesize\ttfamily} %
%\lstset{keywordstyle=\pmb}
%\lstset{basicstyle=\footnotesize\ttfamily} %
\lstset{basicstyle=\ttfamily} %
\lstset{keywordstyle=\bfseries}
\lstset{language=C}
\lstset{mathescape}

\lstset{escapeinside={<@}{@>}}




\begin{document}

\begin{frame}
  \Large

  \begin{center}
    C functional correctness verification comparison\\
    (C-verif-mark?)

    \


    {\large     Peter Sewell \\
      University of Cambridge }

    \

    {\tiny    2022-08-03 Isaac Newton Institute, Verified Software (VSO2) }
  \end{center}
\end{frame}




\begin{frame}{C functional correctness verification}
Crucial -- especially for security-critical systems code

  \

  Still a major research problem -- despite many impressive projects, it's still much harder and more limited than we would like

  \

\pause
  
Hard to understand the state of the art:
\begin{itemize}
\item there are almost as many approaches as there are verification projecs
\item the papers don't always give a clear view of their current strengths, limitations, and future plans,
\item ...or an overview of the design challenges and best solutions.
\end{itemize}

\pause

So, can we establish a better comparison, and shared understanding of the verification-tool design space and alternatives?

\end{frame}


\begin{frame}{Previous comparisons}

  Automated tools:  many very successful  (SV-COMP,...)

  \

  Interactive tools:
  \begin{tightitemize}
 
\item \myhref{A benchmark for C program verification}{https://www.cs.ru.nl/~freek/cbench/cbench.pdf}
\item \myhref{lets-prove-leftpad}{https://github.com/hwayne/lets-prove-leftpad}.
\item \myhref{VerifyThis}{https://www.pm.inf.ethz.ch/research/verifythis.html}, 2011-2021. 
\item \myhref{The 2nd Verified Software Competition: Experience Report}{https://hal.inria.fr/hal-00798777/document}, 2011. 
\item \myhref{The 1st Verified Software Competition: Extended Experience Report}{}
  \end{tightitemize}
  \pause
  Not quite hitting the spot?
\end{frame}

\begin{frame}{Goals}
  \begin{itemize}
  \item lightweight (shared github repo, no separate assessment)
  \item aiming to expose a clear comparison in this multidimensional space \\(not a ``competition'' -- no judging, scores, or winners)
  \item aiming for better understanding of how the different approaches vary
  \item aiming to stimulate future research
  \end{itemize}
\end{frame}


\begin{frame}{Mechanisms}
  \begin{itemize}
\item these talks
  \item smallish collection of smallish examples
    \begin{itemize}
    \item exercising various C language features and programming idioms
    \item small enough for solutions to not be too much work
    \item ask (for at least some) for the most \emph{instructive} solution, not the shortest, and for detailed explanations of what's going on under the hood
    \end{itemize}
  \item larger list of C language features and programming idioms
  \item one or two large examples to focus on scaling
  \item (ideally, ultimately) consensus list of main design challenges and options
  \end{itemize}
\end{frame}
    

  
  


\end{document}
